\section{Limitation and Future Works} \label{sec:fut}
\textbf{Addressing Distribution Independence}
The convolution of two probability distributions assumes that the two distributions are \textit{independent}
of each other.
While we accept this as a limitation of our model, latency in real networks (particularly the latency of two 
queues in the network) might be dependent of each other.

Queuing theory has laid some ground work to describe the asymptotic behavior of such queuing network.
However, they assume some information about the incoming traffic and can only describe a network with a 
particular traffic or delay distribution, like poisson.

Thus, eliminating the independence assumption without demanding an unwieldy amount of information from 
the user while maintaining (or even improving) the accuracy of the model remains an open challenge and an 
interesting future direction.

\textbf{Optimizing Functional Verification}
In our verifier, we built our temporal verifier on top of an existing functional verifier design.
While it makes the design of the verifier simpler, since there is clear separation of concern, there 
might be some performance benefit to gain if we integrate these two parts further.

While we have proposed and evaluated optimization techniques for the temporal verification step, 
integrating this optimization to the functional verification step -- perhaps by consolidating the 
equivalence classes before it is brought to the temporal verification stage in some way -- might make 
our approach more scalable.

While our evaluation shows that the functional verification stage is not the bottleneck of our design, 
discovering the symmetry in the exploration state that might accomplish this task might bring an interesting 
insight to the network verification community in general.