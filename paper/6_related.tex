\section{Related Works} \label{sec:rel}
\textbf{Deterministic Data and Control Plane Verifier}
There has been a substantial body of work concerning the functional behavior of routers
with a given Forwarding Information Base (FIB) or Routing Information Base (RIB).

Prior works like HSA \cite{hsa}, NetKAT \cite{netkat}, and VeriFlow \cite{veriflow} focuses on verifying the 
current data plane forwarding behavior against some property.
While useful in some real-time cases, these tools are not built to explore the many
possible forwarding table given an enumeration of failure cases.

Control plane verifiers such as ARC \cite{arc}, Minesweeper \cite{minesweeper}, and 
Tiramisu \cite{tiramisu} focuses on verifying the various data plane states that 
could be produced by a control plane configuration over multiple failure scenarios.
Many researchers has built on top of this idea, such as DNA \cite{dna} where they 
focuses on verifying the change in property in the event of configuration update.

As comprehensive as it is, both approaches only considers the functional behavior 
of the network (e.g. reachability, loop detection) and they traditionally verify 
the stated properties in a black or white fashion, only determining whether a given 
property is violated or not.

As a result, unlike Tempus, its usefulness does not extend into the use cases of 
verifying SLA, where such binary guarantees are often rightfully avoided in favor 
of a probabilistic and/or quantitative agreement.

\textbf{Probabilistic Verifier}
More recent works have tried to address this very issue.
Works like ProbNetKAT \cite{probnetkat} (a probabilistic verifier based on NetKAT 
\cite{netkat}), NetDice \cite{netdice}, and SRE \cite{sre} provides 
the user ability to define probabilistic failure model, where a given component 
of the network can fail with a given probability.

However, most of them still considers the functional property of the network as the 
main focus. 
While they are useful in verifying certain types of SLA (e.g. availability), they don't 
focus on more quantitative aspect of the network (e.g. bandwidth, latency).

\textbf{Quantitative Verifier}
Some recent works are an exception to this trend, however.
Works like QARC \cite{qarc} (a quantitative verifier based on 
ARC \cite{arc}) has developed a framework to verify one particular quantitative 
property: excess bandwidth on a certain link.
While QARC is a deterministic verifier, like Tempus, they focuses on a more quantitative 
property. 
Unlike Tempus however, this work is orthogonal to verifying latency and thus complementary 
to Tempus.

%TODO: ICNA paper?

\textbf{Network Calculus}
CCAC?
