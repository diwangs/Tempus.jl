\section{Evaluation} \label{eval}

We implemented Tempus in $\sim600$ lines of Julia \cite{julia} code.
This code will take a topology configuration, source and destination node, and the 
temporal property in question and its threshold.
It will then output the final probability of the network upholding that temporal property.
We ran our prototype on a machine with 2 Intel E5-2630 v3 8-core CPU running up to 2.4 GHz 
and 128 GB of ECC RAM, provided by CloudLab \cite{cloudlab}.

To appraise the viability of our approach, we seek to answer 5 evaluation questions:
\begin{enumerate}
    \item How long does it take to run the verifier overall? How does it scale?
    \item What is the bottleneck step in the verification process?
    \item How effective is the optimization technique in reducing the verification time?
    \item How effective is the optimization technique in reducing the explored equivalence classes?
    \item How does different topologies affect optimization effectiveness?
\end{enumerate}

\subsection{Experiment Setup}

\begin{table}[h]
    \begin{tabular}{| c | c | c | c |}
    \hline
    Topology & \#routers & \#links & type \\
    \hline
    Fat Tree 4 & 20 & 32 & DC \\
    Fat Tree 6 & 45 & 108 & DC \\
    Fat Tree 8 & 80 & 256 & DC \\
    Fat Tree 10 & 125 & 500 & DC \\
    Fat Tree 12 & 180 & 864 & DC \\
    Fat Tree 14 & 245 & 1372 & DC \\
    Fat Tree 16 & 320 & 2048 & DC \\
    Fat Tree 24 & 720 & 6912 & DC \\
    Fat Tree 32 & 1280 & 16384 & DC \\
    Latnet & 69 & 74 & WAN \\
    Highwinds & 18 & 31 & WAN \\
    AT\&T & 25 & 56 & WAN \\
    Uninett & 74 & 101 & WAN \\
    GtsCe & 149 & 193 & WAN \\
    \hline
\end{tabular}
\caption{List of topologies being tested}
\label{tab:topologies}
\end{table}

We tested Tempus on 5 real-world WAN topologies from the Topology Zoo and 3 datacenter topologies 
(fat-tree, with various amount of pods).
%TODO: stats of the topologies

\subsubsection{Functional Verification}
As verifying the routing protocol functional behavior is not the primary contribution of our research, 
we simply use OSPF as the routing protocol in our experiment with uniform link weights of 1.
We set 0.1\% failure rate in all links, in accordance to the failure rate in previous studies. %TODO: cite
We then run the functional verification with $10^{-8}$ inaccuracy level.

In order for our evaluation to be realistic, we set the source and destination node to be one of the 
edge routers (node with the smallest degree / smallest centrality?).
%TODO: edge-to-edge routers

\subsubsection{Temporal Verification}
For latency distribution, we show that we can use an empirical measurement as a distribution by 
using the reported queue length from DCTCP \cite{dctcp}.
By multiplying the queue length distribution with the line-rate, we could approximate the queuing 
latency distribution for a given router. 

Since the CDF operation that we define for temporal property verification is relatively cheap, 
we expect our result to generalize with any threshold we chose. 
In other words, while changing the threshold will change the final probability, it won't change 
the runtime performance of Tempus.
Thus, we set an arbitrary threshold for our evaluation.

\subsection{Runtime Profiling}

\begin{figure}[h]
    \centering
    \includegraphics[scale=0.5]{scalability}
    \caption{Runtime performance of the verification process, red dots represent the number of convolutions}
    \label{fig:scalability}
\end{figure}

\begin{figure}[h]
    \centering
    \includegraphics[scale=0.5]{scalability_fattree}
    \caption{Fat Tree, red dots represent the number of convolutions}
    \label{fig:scalabilityfat}
\end{figure}

\subsubsection{Performance and Scalability}
To begin our evaluation, we first measured the running duration of the verification process (and its steps) 
on various topologies.
Our results in Fig. \ref{fig:scalability} shows that our verification technique finished within a reasonable timeframe.
The verifier finished in the order of minutes, even for topologies with hundred of nodes.

From the same result, we also note that our verifier scales gracefully over the size of the network.
The verification duration ranges from 6 seconds to around 15 minutes. 
% TODO: fat-tree scalability
To drive this point more precisely, we also evaluate the running time of fat-tree topology in various sizes.
Our results in Fig. \ref{fig:scalabilityfat} shows that by keeping the same type of topology and scaling them up, we increased 
our verification time almost linearly, while the unoptimized version timed out after 2 hours.
Not only that, our combined optimization strategies actually reduces the time for fat tree topology of $k=10$ compared to 
equivalent topologies with smaller size.

\subsubsection{Bottleneck}
By looking at the proportion of time spent on each step in Fig. \ref{fig:scalability}, we could see that 
the combined \textbf{convolution procedure is the bottleneck} step in our verifier.
The convolution procedure takes 95\% - 99\% of the duration of the overall verification.
Hence we can see in the same figure that the runtime performance of Tempus and the total amount 
of convolution is highly correlated.

We conclude that for a reasonably low imprecision level, \textbf{Tempus runs in the order of 
minutes} and the performance of Tempus is primarily \textbf{bottlenecked by the total convolution 
operations}.

\subsection{Optimization Effectiveness}

\begin{figure}[h]
    \centering
    \includegraphics[scale=0.5]{optimization}
    \caption{Amount of convolution (in log scale) depending on what optimization strategy is applied}
    \label{fig:opt}
\end{figure}
\begin{figure}[h]
    \centering
    \includegraphics[scale=0.5]{optimization_fattree}
    \caption{Amount of convolution (in log scale) depending on what optimization strategy is applied. 
        Fattree 12 upwards is extrapolatied from the number of convolution}
    \label{fig:opt_fat}
\end{figure}

We have established that the convolution procedure is a relatively expensive operation within 
our verifier.
Next, we inspected the effectiveness of our optimization techniques in reducing them.
To do that, we measured the overall running duration of the verifier while selectively disabling the 
optimization.

Our results in Fig. \ref{fig:opt} shows the effect of the two optimization techniques we introduced -- 
consolidation and memoization -- in reducing the verification duration and amount of convolution (both in log scale).
For baseline, we ran the verifier without any optimization, resulting in the left bar.
The middle bar represents the result where we enable only the consolidation strategy.
Finally, the right bar represents the result where we enable both the consolidation and memoization strategy.

We note that for most of the topologies, the combination of both optimization strategies resulted in 
\textbf{79\% - 98\% improvement in performance}.
Compared to the baseline performance, the consolidation strategy contributes to 30\% - 94\% of the improvement and
the memoization strategy contributes to 67\% - 96\% on top of that. 
%TODO: factors of the network node-pair? path amount and length?

We conclude that \textbf{consolidation and memoization are both effective optimization strategies} in our verification
framework.

\subsection{Equivalence Class Reduction}

\begin{figure}[h]
    \centering
    \includegraphics[scale=0.5]{ec}
    \caption{Amount of temporal EC compared to functional EC, label represents ratio}
    \label{fig:ec}
\end{figure}

While we cannot directly compare Tempus with other verifiers (due to difference in verification goal), 
we could use the amount of equivalence class as an indirect proxy of the verifier's behavior and performance.
Therefore, we measured the amount of additional equivalence classes introduced by Tempus in order to 
indirectly compare its overhead in addition to its functional counterparts.
This result is only influenced by the consolidation strategy, since the memoization strategy operates on 
a smaller granularity than equivalence classes.

Our results in Fig. \ref{fig:ec} shows the ratio between the amount of temporal equivalence classes that is 
being re-explored and the amount of functional equivalence class.
We note that we only need to \textbf{re-explore 4\% to 46\% equivalence classes} as an additional overhead 
compared to functional verification.

We conclude that \textbf{consolidation is effective in reducing the amount of equivalence classes that 
needs to be re-explored} in our verification framework.

\subsection{Optimization Analysis}

\begin{table}[h]
    \begin{tabular}{| c | c | c | c | c |}
        \hline
        Top. & \# path & \# path (cons) & \# path (base) & \# conv \\
        \hline
        % Latnet & 0\% & 4.2 \\
        % Highwinds & 97.7\% & 581.41  \\
        % AT\&T & 98.4\% & x \\
        % Uninett & 82.5\% & 108.87 \\
        % GtsCe & 98.2\% & 571.14 \\
        ft4 & 20 & 92 & 251 & 204 \\
        ft6 & 81 & 1,278 & 7,857 & 855 \\
        ft8 & 304 & 42,892 & 500,195 & 3280 \\
        ft10 & 25 & 456,700 & 6,669,169 & 175 \\
        ft12 & 36 & 14,367,491 & 288,537,289 & 252 \\
        ft14* & 49 & 23,767,482 & 323,686,008 & 343 \\
        ft16* & 64 & 21,272,060 & 261,254,848 & 448 \\
        ft24* & 144 & 19,064,581 & 132,197,345 & 1008 \\
        ft32* & 256 & 15,926,575 & 75,783,276 & 1792 \\
        \hline
    \end{tabular}
    \caption{Effectiveness metric}
    \label{tab:analysis}
\end{table}

Up to this point, we have demonstrated that consolidation and memoization are two effective strategies in 
reducing the verification duration and amount of convolution procedure that needs to be done.
However, the results also show that their effectiveness varies depending on the topology.

\subsubsection{Why is Fat Tree 10 better than Fat Tree 4?}
To analyze what affects the effectiveness of our optimization technique, we will start by explaining 
one peculiar result in our evaluation: despite having a bigger size, the temporal verification runtime 
of fat tree with $k = 10$ is faster than $k = 4$.

This is due to the fact that the shortest path in $k = 10$ is a lot more diverse than that of $k = 4$.
In a 3-tier Fat Tree topology, an edge node could reach another pod's edge node in at least 4 hops.
The amount of this shortest path is $(k/2)^2$, which is the same as the number of core node.

We could see from Table \ref{tab:analysis} that when $k \geq 10$, the amount of unique path explored by the 
functional verifier is exactly the same as the number of core node. 
Combined with the fact that the average amount of convolution per path is 7 (which suggests that the path 
length is 4), means that the functional verifier only produces equivalence classes that consisted of only 
different combination of these shortest paths.

When $k < 10$ however, the functional verifier also needs to produce equivalence classes that results 
in a longer convergent paths in order to reach the same accuracy level.
in Fat Tree, this is usually marked by a visit to another aggregation node in another pod. 
We could see from Table \ref{tab:analysis} that when $k < 10$, the amount of of convolution per path is more 
than 7 (which suggests that a path longer than 4 hops exist).

\subsubsection{Generalization}
From the insight of this particular result, we could draw two properties in a general network that could 
determine the runtime of our verifier.

The first one is the \textbf{amount of possible paths}.
If the amount of possible paths between a src-dst pair in a given topology is small, then the functional 
verification step would produce fewer equivalence classes and end early.
While most of our evaluated topology had a lot of possible paths, one exception to this is LatNet, which 
only have 3 possible paths.

The second is whether the src-dst pair in a given topology would \textbf{produce paths that are "symmetric"}.
By symmetric, we mean that the routing and load balancing method produce a lot of identical equivalence class 
and / or equivalence class that share the same path.
As our explanation about Fat Tree 10 suggests, larger Fat Tree has a lot of path with the same length.
This would result in a lot of symmetrical equivalence classes under ECMP.
Our optimization techniques could identify this symmetry and reduce the amount of convolution in the temporal 
verification step.

%TODO: path enumeration?