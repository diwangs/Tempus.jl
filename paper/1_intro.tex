\section{Introduction}
The intricate configuraton of network devices is notoriously hard to get right, forcing network 
engineers to accept the reality of Murphy's Law or getting some help from a tool to ensure their 
correctness, namely, network verifier.
Over the last decade, there has been a lot of development in the area of network verification 
generally \cite{hsa}\cite{veriflow}.
While their approach varies a lot (e.g. data plane vs. control plane verification, deterministic vs. 
probabilistic), most of these verifiers shares a common focus, primarily fixated on verifying 
\textit{qualitative} properties -- the convergent functional behavior of a network under various failure 
scenarios -- such as reachability and loop detection.

However, network engineers also have the need to verify \textit{quantitative} properties, such as properties 
related to bandwidth and latency.
Still, barring some recent works about the verification of link load violation \cite{qarc}, work in this area 
remain sparse, despite its undoubtable importance.

In this work, we're exploring the verification of the other side of the network performance metric: latency. 
Certain network deployment often necessitates some latency requirement such as an ISP that has 
latency SLA \cite{Verizon} or deployments of Time-Sensitive Networking (TSN) \cite{TSN}.
We proposed a verification framework to probabilistically verify the latency property of packets 
traversing from a source to destination node under various failure scenarios, by using latency 
measurements of the components in the network.

We introduce the design and implementation of \tool, a probabilistic network latency verification 
framework.
\tool will use the latency information from relevant component measurements (e.g. router queue length) 
to infer the latency distribution of said component.
Assuming that future traffic is i.i.d., we then employ a numerical convolution method to combine the
latency distribution of each relevant components together to produce the end-to-end latency distribution 
of an src-dst router pair. 

To determine said relevant components, we built \tool on top of an existing qualitative verification 
framework to efficiently explore the path used for an src-dst router pair given various scenarios of 
failures.
By obtaining the end-to-end latency distribution from this information and the latency measurement, 
we could analyze the statistical properties of the src-dst pair latency distribution to probabilistically 
argue about the latency properties.

We also introduced two optimization techniques on top of this framework to reduce the verification time 
by multiple orders of magnitude.
These two optimization techniques rely on the symmetry that we have found in the quantitative verifier 
when brought into the context of latency verification.

Our evaluation shows that the verification framework, with the help of our optimization techniques, 
could accomplish the verification task on various WAN and datacenter topologies in the order of 
minutes?

With this work, we make the following technical contributions:
\begin{itemize}
    \item \textbf{Novel temporal verification framework} using numerical convolution of network 
        components' latency measurement.
    \item \textbf{Two optimization techniques} in said verification framework by exploiting 
        symmetry in the failure exploration.
    \item \textbf{\tool}, an implementation of our verification framework and its optimization 
        on Julia.
\end{itemize}

The code for this work is open-sourced at link. %TODO: change link
This work does not raise any ethical issues.

\begin{figure*}[h]
    \centering
    \includegraphics{../tikz/process}
    \caption{Overview of Tempus}
    \label{fig:process}
\end{figure*}