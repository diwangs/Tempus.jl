\section{Introduction}
Modern networks consisted of various distributed protocols such as OSPF and BGP that exchange 
routing information so that a packet can reach their intended destination efficiently, even in the 
event of a component failure.
Due to their configuration intricacies however, these protocols are notoriously hard to get right. 
It is hard for a network engineer to reason about whether their configured network fulfilled some 
intended property in various possible states of the network. 
Thus, they are left between the choice of accepting the reality of Murphy's Law or getting a tool 
to help them in this task, namely, network control plane verifier.

Popular control plane verifiers, like ARC \cite{arc}, are formulated to answer 
questions about a given property deterministically. 
In other words, given a set of states of the network (e.g. $k$-link failures) the verifier are 
expected to give a yes or no answer about the property (e.g. two nodes are reachable).
While useful to some extent, this kind of models are sometimes too restrictive since network 
operators are usually able to tolerate small fraction of failure. 
For example, a given network might have an availability SLA of 99.999\%. 
This kind of probabilistic properties have been the focus of a more recent works like NetDice 
\cite{steffen2020probabilistic}.

The network property itself can also be divided into two kinds. 
Quantitative properties like reachability and loop existence, are the common properties that are 
studied by most of the existing verifiers. 
More recent work, like QARC \cite{subramanian2020detecting}, has also explored qualitative 
properties like link bandwidth violation for a given traffic.

In this work, we're exploring the other side of the network performance metric: latency. 
Certain network deployment often necessitates some latency requirement such as an ISP that has 
latency SLA \cite{Verizon} or deployments of Time-Sensitive Networking (TSN) \cite{TSN}.
We proposed a verification framework to probabilistically verify the latency property of packets 
traversing from a source to destination node under various failure scenarios, by using latency 
information of each components in the network.

\begin{figure*}[h]
    \centering
    \includegraphics{../tikz/process}
    \caption{Overview of Tempus}
    \label{fig:process}
\end{figure*}