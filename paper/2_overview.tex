% \section{Overview}
\section{Latency Modeling Background}
In the context of verification, there are multiple approaches to model latency in a network.
We will briefly compare 4 possible methods that might be useful to attack this problem and 
and its limitation to illustrate the methodological challenge to accurately model latency.

\subsection{Simulation}
One of the available methods to reason about latency in a network is by simulating it 
using a network simulator (e.g. ns3, omnet).
On a high level, the tool will simulate the lifecycle of packets, created by a load generator, 
and the user could fail some network components and collect the packets' latency measurement 
to see how the perturbation affects the latency property in question.

Network simulator has been widely used in our community as an alternative to experimenting on 
a bare-metal system. 
While generally accurate, network simulators notoriously take a long time to run.
Moreover, the simulation workflow is more akin to testing than verification, relying on 
the user to create failure scenarios and reason about latency in a contrapositive way -- 
the absence of a negative result does not mean the property is guaranteed to be fulfilled.

\subsection{Timed Automata}
Timed Automata is an extension to the classic Finite State Machine (FSM) in automata theory that 
adds a global clock to be used as constraints. 
The runtime will keep a set of global clock that monotonously increases and can be individually 
reset.
We could construct the state machine that constraints each transition with a time limit. 
By having this additional time constraints, we could verify properties in a similar way 
as traditional model checking methods (e.g. using CTL formulas).

While this technique offers a proper verification framework and a higher level of abstraction 
compared to simulation, traditional Timed Automata theory is deterministic, offering no random 
transitions to express random component failure and a way to reason about latency probabilistically.

A popular Timed Automata modeling tool, UPPAAL, tried to alleviate this determinism problem by 
introducing Statistical Model Checking (SMC) on top of their Timed Automata model. 
The feature allows user to introduce random transition (with configurable weights) and distribution 
based transition, in order to probabilistically reason about the desired property and model 
random failures of network components.

However, they did it by collecting the statistical property of multiple simulation runs, which 
has the same drawback as the simulation-based method. 
It is also limited in its choice of distribution, only allowing user to use uniform and 
exponential distribution.

\subsection{Queuing Theory}
Another theoretical framework that is more closely related to networking devices are queuing 
theory.
Queuing theory allows you to reason about the equilibrium latency of a given traffic, given the 
arrival and processing distribution to each component, among other things.

One major downside of this framework, however, is the assumptions about what kind of probability 
distributions being used. 
There is a closed-form solution for a network where the queues are identified as a poisson process 
(i.e. Jackson network). 
Still, a queuing network with a more diverse queuing behavior does not have as strong theoretical 
answer in this framework.

\subsection{Convolution}

% \subsection{Latency as Path Property} 
% We began our study by first pondering about a basic construct: modeling the end-to-end latency of 
% a single packet.
% In a packet-switched network, a packet is sent from one transmiting host to one receiving host 
% through the many components of the network (e.g. links, routers, firewall) and each of those 
% components might introduce some latency into the packet transmission process.
% Figuring out which of those components will actually introduce latency into the packet in question, 
% and by how much, is the next logical step that we must figure out.

% Obviously, a given packet doesn't need to visit all network components to reach its destination 
% host.
% A network engineer will configure the network in such a way that a packet will only need to be 
% routed via a specific subset of its components.
% That specific subset is dictated by two things: how those components are connected together 
% (i.e. topology) and how the control plane protocols are configured to route a packet (e.g. routing 
% protocol, ACL).
% Based on a variety of these setups, nodes in the network will form a forwarding table to route a 
% given packet appropriately.

% The goal of a classical control plane verifier then, is to use this forwarding table in some 
% form to verify certain properties. %TODO: examples?
% However, looking at the forwarding table alone will not give us a conclusive result regarding 
% which components are going to be visited by a given packet, making verification of latency 
% properties less clear.
% For example, the network might be configured with a load-balancing protocol in which a packet 
% departing from a source host might take multiple different \textit{paths} (with certain 
% probability weights) to arrive at the destination host, possibly resulting in a different 
% end-to-end latency measurement.

% Therefore, we argue that when it comes to analyzing end-to-end latency, a network path should be 
% the primary unit of reasoning, rather than forwarding table.
% By being more specific about our unit of reasoning, we could answer verification questions more 
% clearly, and we could design our verifier more efficiently since we could use it to represent 
% multiple different forwarding tables that shares the same path.

% \subsection{Relation to Classical Verifier}
% Before we analyze the latency of a given packet that propagates through a certain network 
% path however, we must make sure that said path exists in the first place.
% We note that latency is a property that only make sense after connectivity between two hosts 
% has been established. 
% In other words, if two nodes in a network aren't even functionally connected (e.g. physical 
% link failure, ACL policy), then the latency between them will \textit{always} be infinite, 
% making the verification task trivial.

% Fortunately, there are a rich body of work in the network verification literature regarding 
% functional reachability under failure. %TODO: cite
% We could then design our verifier on top of an existing classical control plane verifier.
% We use it to verify reachability property, and only if the reachability property is fulfilled, 
% we would verify whether the latency between two hosts fulfilled some additional condition.

% \subsection{Path Latency Distribution}
% Up to this point, we have talked about measuring the latency of a single packet by figuring out 
% its path; analyzing which exact components it has traversed through.
% However, when we try to generalize this framework and ask questions about the latency of multiple 
% packets, it is apparent that the path alone is not a determinative information, as the latency 
% of two packets propagating through the same path might be different due to a multitude of factors.

% The natural extension to the framework then, is instead of representing latency of a path with a 
% single number, we instead represent it with a continuous random variable that signifies the 
% possible delay that a given packet traversing through that path might have.
% This random variable will have a distribution that marginalizes over all other factors other than 
% the path.

% We can then use this latency distribution to verify some temporal properties in a probabilistic 
% manner.
% For example, we could verify the probability that a packet will be delivered in under a time 
% unit by taking the CDF of the distribution.

% \subsection{Path Decomposition and Convolution} \label{decomposition}
% The final question that we had to decide on was how do we actually model path latency distribution
% based on real component measurements.
% Considering the complexity of factors that might determine the path latency distribution and the 
% availability of measurement data, we settle on the assumption that the path latency distribution 
% is composed of multiple independent distribution that corresponds to the latency each components 
% in the path might introduce.
% %TODO: confirm about independence?

% Since not all components in the path will introduce latency with a non-negligible value, we 
% specifically choose to model two source of latency in the path's components which we deem 
% significant: \textbf{link propagation latency} and \textbf{queuing latency}.
% %TODO: expand on how to get the data later

% Propagation delay is the latency that is introduced by the links in the network, which is 
% independent of the traffic load in the system.
% Queuing delay is the latency that the queuing process in the node introduced. 
% Unlike propagation delay, queuing delay might be dependent on load in the system, since the more 
% packets there are in the queue, the more delay the node will introduce to a subsequent packets.

% For propagation delay, the semantic of this random variable is relatively straightforward: it is 
% the distribution of latency that a given link will introduce.
% For queuing delay however, this random variable represents the delay that a given queuing process 
% will introduce, marginalized over various traffic pattern that a given network state might have 
% resulted.

% In order to obtain the overall path latency distribution from these per-component latency 
% distributions, we do a convolution operations over all the relevant component's latency 
% distributions.
% Since not all distributions can be convolved in a closed-form fashion, we use a numerical 
% convolution technique with a guaranteed error bound. 
% We initially consider a Monte-Carlo simulation in order to approach numerical convolution, but 
% the lack of a general technique to measure error made us opt for the formerly mentioned method.


% \section{Verifier Design}

We divide the problem of latency verification into two parts: 
verifying that two nodes are reachable (\textbf{functional verification}) and only if 
the functional property is fulfilled, we would verify whether the latency between 
two nodes fulfilled some temporal condition (\textbf{temporal verification}).