\documentclass[10pt,sigconf,letterpaper,anonymous,nonacm]{acmart}

\usepackage{amsfonts}
\usepackage{graphicx}
\usepackage{tikz}
\usepackage{tikzscale}
\usetikzlibrary{positioning}
\usetikzlibrary{shapes.multipart}
\usetikzlibrary{graphs}
\usetikzlibrary{trees}
\usetikzlibrary{quotes}

\graphicspath{ {../viz/} }

\title{Tempus: Probabilistic Network Latency Verifier}
\author{WIP}

\begin{document}
\begin{abstract}
    To combat problems and bugs that a given network might have, network verifiers have emerged as 
    one of the promising solutions. 
    State-of-the-art network verifiers mainly focused on evaluating qualitative properties under 
    various scenarios of network failures, such as reachability under k-link failure. 
    However, as modern networks evolved and performance need becomes more stringent -- often 
    expressed in terms of Service Level Agreement (SLA) -- there is a need to evolve network 
    verifiers to also reason about quantitative performance properties. 
    Works in quantitative network verifiers that has arose in recent years mainly focused on one 
    side of the network performance metric: bandwidth and load violation properties. 
    Questions about the other side of network performance metric, latency, were left unanswered. 

    In this work, we introduce a verifier framework, Tempus, that can answer the probability of 
    a given temporal property being true given latency distributions of individual links and 
    nodes in the network. 
    Early evaluation shows that Tempus can verify bounded reachability property -- the probability
    that the average delay of packets traversing from a source to destination node is below $T$
    time unit -- by only adding a fraction of the state exploration overhead introduced by the 
    qualitative verifier it's built upon.
\end{abstract}

\maketitle

\section{Introduction}
Modern networks consisted of various distributed protocols such as OSPF and BGP that exchange 
routing information so that a packet can reach their intended destination efficiently, even in the 
event of a component failure.
Due to their configuration intricacies however, these protocols are notoriously hard to get right. 
It is hard for a network engineer to reason about whether their configured network fulfilled some 
intended property in various possible states of the network. 
Thus, they are left between the choice of accepting the reality of Murphy's Law or getting a tool 
to help them in this task, namely, network control plane verifier.

Popular control plane verifiers, like ARC \cite{arc}, are formulated to answer 
questions about a given property deterministically. 
In other words, given a set of states of the network (e.g. $k$-link failures) the verifier are 
expected to give a yes or no answer about the property (e.g. two nodes are reachable).
While useful to some extent, this kind of models are sometimes too restrictive since network 
operators are usually able to tolerate small fraction of failure. 
For example, a given network might have an availability SLA of 99.999\%. 
This kind of probabilistic properties have been the focus of a more recent works like NetDice 
\cite{steffen2020probabilistic}.

The network property itself can also be divided into two kinds. 
Quantitative properties like reachability and loop existence, are the common properties that are 
studied by most of the existing verifiers. 
More recent work, like QARC \cite{subramanian2020detecting}, has also explored qualitative 
properties like link bandwidth violation for a given traffic.

In this work, we're exploring the other side of the network performance metric: latency. 
Certain network deployment often necessitates some latency requirement such as an ISP that has 
latency SLA \cite{Verizon} or deployments of Time-Sensitive Networking (TSN) \cite{TSN}.
We proposed a verification framework to probabilistically verify the latency property of packets 
traversing from a source to destination node under various failure scenarios, by using latency 
information of each components in the network.

\begin{figure*}[h]
    \centering
    \includegraphics{../tikz/process}
    \caption{Overview of Tempus}
    \label{fig:process}
\end{figure*}

% \section{Overview}
\section{Latency Modeling Background}
In the context of verification, there are multiple approaches to model latency in a network.
We will briefly compare 4 possible methods that might be useful to attack this problem and 
and its limitation to illustrate the methodological challenge to accurately model latency.

\subsection{Simulation}
One of the available methods to reason about latency in a network is by simulating it 
using a network simulator (e.g. ns3, omnet).
On a high level, the tool will simulate the lifecycle of packets, created by a load generator, 
and the user could fail some network components and collect the packets' latency measurement 
to see how the perturbation affects the latency property in question.

Network simulator has been widely used in our community as an alternative to experimenting on 
a bare-metal system. 
While generally accurate, network simulators notoriously take a long time to run.
Moreover, the simulation workflow is more akin to testing than verification, relying on 
the user to create failure scenarios and reason about latency in a contrapositive way -- 
the absence of a negative result does not mean the property is guaranteed to be fulfilled.

\subsection{Timed Automata}
Timed Automata is an extension to the classic Finite State Machine (FSM) in automata theory that 
adds a global clock to be used as transition constraints. 
The runtime will keep a set of global clock that monotonously increases and can be individually 
reset.
Using these clocks, we could construct the state machine that constraints each transition with 
a time limit. 
With this additional time constraints, we could verify properties in a similar way 
as traditional model checking methods (e.g. using CTL formulas).

While this technique offers a proper verification framework and a higher level of abstraction 
compared to simulation, traditional Timed Automata theory is deterministic, offering no random 
transitions to express random component failure and a way to reason about latency probabilistically.

A popular Timed Automata modeling tool, UPPAAL, tried to alleviate this determinism problem by 
introducing Statistical Model Checking (SMC) on top of their Timed Automata model. 
The feature allows user to introduce random transition (with configurable weights) and distribution 
based transition, in order to probabilistically reason about the desired property and model 
random failures of network components.

However, they did it by collecting the statistical property of multiple simulation runs, which 
has the same drawback as the simulation-based method. 
It is also limited in its choice of probability distribution, only allowing user to use uniform and 
exponential distribution.

\subsection{Queuing Theory}
Another theoretical framework that is more closely related to networking devices are queuing 
theory.
Queuing theory allows you to reason about the equilibrium latency of a given traffic, given the 
arrival and processing distribution to each component, among other things.

One major downside of this framework, however, is the limitation of assumed probability 
distributions that is being used in each queue and the arrival process. 
Where one could derive a closed-form solution for a network where the queues and / or the arrival process 
are identified as a poisson process (i.e. Jackson network, BCMP network \cite{bcmp}), 
the same thing cannot be said for a queuing network that posits a more diverse queuing behavior.
A framework to describe a queuing network with an arbitrary queuing behavior and arrival process in this 
theory is yet to be discovered.

\subsection{Latency as Random Variable}
% Verification framework that is
% - Expressive (congestion control, load-balancing scheme, traffic generation -> probabilistic)
% - Performant (verification time that makes sense, proper-level of abstraction)
% - Practical (reasonably accurate while requiring minimal amount of information and assumption)

Considering these available approaches and their respective strength and limitation, we devise a 
simple yet powerful framework to verify the latency property in a given network.

Our main idea is to model the latency of a given network component as an independent random variable.
By modeling the latency this way, we could then operate on those random variable to model the latency 
of a given path, and eventually the latency of a given src-dst pair.
Using the resulting random variable, we could then determine the temporal property in question by 
analyzing the statistical property of the resulting probability distribution.

\textbf{This framework is expressive}.
The packet interarrival time that is represented by the latency distribution in each router encodes the 
latency information of the various traffic that is going through the network.
At the same time, it also encodes the effect of congestion and flow control scheme that affects 
those traffic.

\textbf{This framework is performant}.
Operating on the high-level of abstraction of the latency distribution in each network components allows 
this framework to perform the verification task in a matter of minutes?
We demonstrate our evaluation result on Section \ref{eval}.

\textbf{This framework is practical}.
Queue length is a common metric that is used to measure the resource utilization of a network 
\cite{dctcp} \cite{swift}.
Using this metric, one could easily construct a latency distribution by combining it with the 
line rate bandwidth.

% \subsection{Latency as Path Property} 
% We began our study by first pondering about a basic construct: modeling the end-to-end latency of 
% a single packet.
% In a packet-switched network, a packet is sent from one transmiting host to one receiving host 
% through the many components of the network (e.g. links, routers, firewall) and each of those 
% components might introduce some latency into the packet transmission process.
% Figuring out which of those components will actually introduce latency into the packet in question, 
% and by how much, is the next logical step that we must figure out.

% Obviously, a given packet doesn't need to visit all network components to reach its destination 
% host.
% A network engineer will configure the network in such a way that a packet will only need to be 
% routed via a specific subset of its components.
% That specific subset is dictated by two things: how those components are connected together 
% (i.e. topology) and how the control plane protocols are configured to route a packet (e.g. routing 
% protocol, ACL).
% Based on a variety of these setups, nodes in the network will form a forwarding table to route a 
% given packet appropriately.

% The goal of a classical control plane verifier then, is to use this forwarding table in some 
% form to verify certain properties. %TODO: examples?
% However, looking at the forwarding table alone will not give us a conclusive result regarding 
% which components are going to be visited by a given packet, making verification of latency 
% properties less clear.
% For example, the network might be configured with a load-balancing protocol in which a packet 
% departing from a source host might take multiple different \textit{paths} (with certain 
% probability weights) to arrive at the destination host, possibly resulting in a different 
% end-to-end latency measurement.

% Therefore, we argue that when it comes to analyzing end-to-end latency, a network path should be 
% the primary unit of reasoning, rather than forwarding table.
% By being more specific about our unit of reasoning, we could answer verification questions more 
% clearly, and we could design our verifier more efficiently since we could use it to represent 
% multiple different forwarding tables that shares the same path.

% \subsection{Relation to Classical Verifier}
% Before we analyze the latency of a given packet that propagates through a certain network 
% path however, we must make sure that said path exists in the first place.
% We note that latency is a property that only make sense after connectivity between two hosts 
% has been established. 
% In other words, if two nodes in a network aren't even functionally connected (e.g. physical 
% link failure, ACL policy), then the latency between them will \textit{always} be infinite, 
% making the verification task trivial.

% Fortunately, there are a rich body of work in the network verification literature regarding 
% functional reachability under failure. %TODO: cite
% We could then design our verifier on top of an existing classical control plane verifier.
% We use it to verify reachability property, and only if the reachability property is fulfilled, 
% we would verify whether the latency between two hosts fulfilled some additional condition.

% \subsection{Path Latency Distribution}
% Up to this point, we have talked about measuring the latency of a single packet by figuring out 
% its path; analyzing which exact components it has traversed through.
% However, when we try to generalize this framework and ask questions about the latency of multiple 
% packets, it is apparent that the path alone is not a determinative information, as the latency 
% of two packets propagating through the same path might be different due to a multitude of factors.

% The natural extension to the framework then, is instead of representing latency of a path with a 
% single number, we instead represent it with a continuous random variable that signifies the 
% possible delay that a given packet traversing through that path might have.
% This random variable will have a distribution that marginalizes over all other factors other than 
% the path.

% We can then use this latency distribution to verify some temporal properties in a probabilistic 
% manner.
% For example, we could verify the probability that a packet will be delivered in under a time 
% unit by taking the CDF of the distribution.

% \subsection{Path Decomposition and Convolution} \label{decomposition}
% The final question that we had to decide on was how do we actually model path latency distribution
% based on real component measurements.
% Considering the complexity of factors that might determine the path latency distribution and the 
% availability of measurement data, we settle on the assumption that the path latency distribution 
% is composed of multiple independent distribution that corresponds to the latency each components 
% in the path might introduce.
% %TODO: confirm about independence?

% Since not all components in the path will introduce latency with a non-negligible value, we 
% specifically choose to model two source of latency in the path's components which we deem 
% significant: \textbf{link propagation latency} and \textbf{queuing latency}.
% %TODO: expand on how to get the data later

% Propagation delay is the latency that is introduced by the links in the network, which is 
% independent of the traffic load in the system.
% Queuing delay is the latency that the queuing process in the node introduced. 
% Unlike propagation delay, queuing delay might be dependent on load in the system, since the more 
% packets there are in the queue, the more delay the node will introduce to a subsequent packets.

% For propagation delay, the semantic of this random variable is relatively straightforward: it is 
% the distribution of latency that a given link will introduce.
% For queuing delay however, this random variable represents the delay that a given queuing process 
% will introduce, marginalized over various traffic pattern that a given network state might have 
% resulted.

% In order to obtain the overall path latency distribution from these per-component latency 
% distributions, we do a convolution operations over all the relevant component's latency 
% distributions.
% Since not all distributions can be convolved in a closed-form fashion, we use a numerical 
% convolution technique with a guaranteed error bound. 
% We initially consider a Monte-Carlo simulation in order to approach numerical convolution, but 
% the lack of a general technique to measure error made us opt for the formerly mentioned method.


% \section{Verifier Design}

We divide the problem of latency verification in this framework into two parts: 
verifying that two nodes are reachable (\textbf{functional verification}) and only if 
the functional property is fulfilled, we would verify whether the latency between 
two nodes fulfilled some temporal condition (\textbf{temporal verification}).
In Section \ref{sec:functional}, we briefly describe the functional verification scheme that 
our framework is built upon. 
Next, in Section \ref{sec:temp}, we describe our novel temporal verification framework 
in more detail.
Next, in Section \ref{sec:opt}, we describe the optimization techniques that arise from 
these two steps.
We then evaluate our implementation in Section \ref{eval}.
Finally, we will touch on some related works on Section \ref{sec:rel}, limitations and future 
works on \ref{sec:fut}, and conclude our work with Section \ref{sec:conc}.

\section{Functional Verification} \label{sec:functional}
For functional property verification, NetDice \cite{netdice} 
has laid the way for verifying reachability between two nodes under failure in a 
probabilistic and efficient manner.
To contextualize our modification, we will briefly describe some parts of 
NetDice's network encoding and exploration algorithm as a prelude to our alteration.
For a more complete complete description of NetDice, please refer to the original 
paper.

\begin{figure}[h]
    \centering
    % \includegraphics[width=0.4\textwidth, trim=0cm 11cm 0cm 5cm]{ex.pdf}
    \includegraphics{../tikz/ex}
    % \includegraphics{ex.eps}
    \caption{Example topology, we want to check the reachability of A to Z}
    \label{fig:ex}
\end{figure}

\subsection{Topology Graph}
In this framework, a network is encoded in an edge-labeled directed graph 
$G_t = (V_t, E_t)$ where $V_t$ represents the nodes in the network and 
$E_t$ represents the functional connectivity between a source and a destination node. 
A physical link is then represented as a pair of symmetrical edges that shares the same source 
and destination node but with opposite direction (e.g. $v_1 \rightarrow v_2$ and $v_2 
\rightarrow v_1$).
A routing protocol is then defined on top of this graph by some auxilary information.

To explain it with further clarity, we use the topology in Fig. \ref{fig:ex} as a 
running example for this paper. 
We will assume that the nodes in the network runs OSPF with weight 1 for every edge and ECMP 
as their load-balancing scheme.
We want to check the temporal probability of packets that departs from A to Z.

% To represent the component's random failure, NetDice defined two failure rates.
% The first failure rate is the chance that a given physical link in the network will go down.
% The second failure rate is the chance that a given node in the network will go down.
% These rates are shared between all links / nodes in the network.
% Internally however, the node failure rate will be translated into the link failure rate 
% with a Bayesian Network model, since a node failure can be modeled with the failure of each 
% links connected to it.

To represent random failure, NetDice defined a universal link failure rate (i.e. 
the probability of \textit{any} link in the network randomly failing).
We refine NetDice's model slightly by allowing each links to have different failure rate.
We label each edge in $E_t$ with a function $r: E_t \rightarrow \{x \in \mathbb{R} \mid 0 \le x 
\le 1\}$ that represents the failure rate of a given physical link.
As a consequence, two symmetrical edges (i.e. two edges that shares the same node pairs but with 
opposite direction) will also share the same failure rate, and will be disabled in a coupled 
fashion.

For the sake of example, in our Fig. \ref{fig:ex} topology, we will assume that each link has a 
10\% chance of failure.

% \subsubsection{Routing}
% On top of this topology, NetDice also defined additional informations that would be used
% by a routing protocol to determine the valid path(s) between two nodes $src, dst \in V_t$ given 
% a particular link failure scenario. 
% These routing informations would also be used by their optimization algorithms to reduce the 
% amount of states that are going to be explored.

% NetDice implemented iBGP and OSPF (with ECMP) routing protocols.
% They encode the relevant routing information by assigning some labels to the vertices or edges
% in the topology.
% In OSPF for example, we define a function $w_{ospf}: E_t \rightarrow \mathbb{N}$ as the edge-label 
% that represents the positive weight of a link.
% They would later be used to compute the convergent paths of a given network state 
% (\ref{verification}).

\subsection{State Exploration}
A \textbf{network state} is defined as a set of failed links.
There are $2^{|E_t|}$ network states in a given network and a brute-force strategy 
would need to explore all of the states to compute the reachability property of a 
node pair in a given network. 
NetDice however, will merge multiple network states into an \textbf{equivalence class}
by marginalizing over \textit{cold edges}, links whose failure is guaranteed not to 
change the convergent path of a highlighted network state.

NetDice will systematically explore many equivalence classes in the network. 
It will start from the equivalence class of a perfect network and will fail certain 
\textit{hot edges} (i.e. edges that are not cold) to explore another equivalence 
class.
While not explicit in its description, NetDice's algorithm will effectively form an 
\textbf{exploration tree}. 

In order to preserve some information that will be used in the temporal verification 
stage, we modify this original exploration algorithm to make it more explicit.
We started by formalizing the notion of Equivalence Classes.
An Equivalence Class $\mathcal{E}$ is defined as a 3-tuple $(U_h, D_h, paths)$.
$U_h$ and $D_h$ refers to a set of hot edges that are up and down respectively.
$paths$ refers to the convergent paths between src-dst pair that is produced by the
control plane when the links in $D_h$ fails.
Unlike the original algorithm, we explicitly store $paths$ in $\mathcal{E}$ so that 
it could be used to for the temporal verification stage.

We then define the Exploration Tree $\mathcal{T}$ as a tree of Equivalence Classes.
At the root, we have an Equivalence Class that corresponds to a perfect network, where 
$paths$ is the path(s) that the control plane will produce given a perfect network 
condition and $U_h = D_h = \emptyset$.
Each Equivalence Class in this tree will have children which have one of its parent's link in 
$paths$ appended to its $D_h$, essentially failing one of the link in the path.
If an Equivalence Class has an empty $paths$, then it would have no children.

To compute the functional probability $P_f$ of a given Equivalence Class, we compute the 
product of the 'up' probability of all the links in $U_h$ and $paths$ and the 
product of the 'down' probability of all the links in $D_h$.

In our running example, we could see the subset of the exploration tree in Fig. 
\ref{fig:tree}. 
We see that in a perfect network ($\mathcal{E}_1$), OSPF will produce $ACZ$ as the 
shortest path between A to Z.  
The probability of this Equivalence Class actually materializing ($P_f$) is $0.81$, 
which is the probability of $AC$ and $CZ$ being up at the same time ($0.9^2$).

In the Equivalence Class where the link $AC$ is failing however ($\mathcal{E}_2$), 
OSPF and ECMP will produce two equaly-weighted paths $ABEZ$ and $ADFZ$.
The probability of this Equivalence Class actually materializing ($P_f$) is $0.053$
($0.9^6 \cdot 0.1$).

After having the algorithm produced $\mathcal{T}$, we then continue to the temporal verification 
stage.

\section{Temporal Verification} \label{sec:temp}
In the temporal verification stage, we will use the path information that we got from 
functional verification stage and augment them with latency information to 
produce the final temporal property probability.

\begin{figure}[h]
    \centering
    % \includegraphics[width=0.4\textwidth, trim=0cm 11cm 0cm 5cm]{ex.pdf}
    \includegraphics[width=\columnwidth]{../tikz/tree}
    % \includegraphics{ex.eps}
    \caption{Subset of Modified Exploration Tree $\mathcal{T}$ for our example topology, 
        $\mathcal{E}_2$ and $\mathcal{E}_3$ will get consolidated}
    \label{fig:tree}
\end{figure}

\subsection{Latency Label}
Similar to link failure $r$, we use edge-labeling technique to represent latency that 
will get introduced by the connectivity between two nodes. 
Our model will model two significant sources of latency: \textbf{link propagation} 
and \textbf{packet queuing} in the node.
We will encode these latencies by equipping the model with two additional labels.

Encoding link propagation latency is fairly straightforward. 
We define a function $l_p: E_t \rightarrow \mathcal{D}$ where $\mathcal{D}$ is a set of 
continuous univariate distribution that has a minimum value of $0$.
This distribution signifies the time it would take for the respective physical link to transmit 
a packet from one end to another.
Because of this, just like $r$, two symmetrical edges will share the same distribution.

To encode queuing latency, we first note that the queuing mechanism in modern switches 
usually resides in the output port. %TODO: Cite PISA?
Since one port in a switch is only connected to one port in another switch, we could effectively 
assign queuing latency to the connectivity between switches.
To do this, we define another function $l_q: E_t \rightarrow \mathcal{D}$.
This distribution signifies the output queue latency in the source node that is 
going to be forwarded to the destination node.
Unlike $l_p$, two symmetrical edges will have two different distributions since they 
represent two different output queues.

% In our running example, we will assume that all edges in the topology will have $l_p$
% and $l_q$ of gamma distribution with $\lambda = 1$, except $l_q(EZ)$ which will 
% have a lognormal distribution with $\alpha = 1$ and $\theta = 1$

\subsection{Latency Measurement as Distribution}
To have the required latency distribution for $l_p$ and $l_q$, we form an empirical distribution 
from latency measurement information.
This measurement information might come in the form of packet inter-arrival time, 
queue length, and other data that is similar in nature.

Since packet inter-arrival time represent the effect of various traffic shaping mechanism on 
the many traffics on the network, we could obtain such data from published works like 
\cite{dctcp}\cite{swift}.
Central to our model is the assumption that the latency introduced by the covered component is 
i.i.d. and are independent of one another.

Given $paths$, $l_p$, and $l_q$, we could then compute $P_t$, the probability of a 
given temporal property being fulfilled.
We will describe our algorithm for computing this probability in the next few 
sections.

\subsection{Weighted Average and Path Convolution}
Since a convergent behavior of the forwarding plane $paths$ might contain more than 
one paths, we will compute the temporal probability of $paths$ by computing the 
weighted average of the temporal probability of each of the individual path.
The specific weight depends on the load-balancing method used.
In our example, we will use ECMP load-balancing scheme.

To get the temporal probability of a single path, we will use the latency information 
from $l_p$ and $l_q$ of each edge in the path to compute the latency distribution of the whole path.
To do this, we resort to the methods of \textbf{convolution}.
For each edge $e$ in the path, we will get $l_p(e)$ and $l_q(e)$ and convolve them 
all together to get the path latency distribution.

With this path latency distribution, we could get the temporal probability of a 
path by computing the statistical property of said distribution.
In this work, given a path latency distribution $\mathcal{L}$, we define two temporal 
properties:
\begin{itemize}
    \item \textbf{Bounded Reachability}: the probability that a packet will get 
        transmited below $t$ time unit. This will get computed as $cdf(\mathcal{L}, t)$
    \item \textbf{Tail Reachability}: the probability that a packet will get 
        transmited above $t$ time unit. This will get computed as 
        $1 - cdf(\mathcal{L}, t)$
\end{itemize}

In our Fig. \ref{fig:tree} Exploration Tree for example, $\mathcal{E}_2$ has two 
equally probable path, $ABEZ$ and $ADFZ$.
We will first do a chain convolution of all the distributions in $ABEZ$ ($l_p(AB)$, 
$l_q(AB)$, ..., $l_q(EZ)$) and do the same thing for $ADFZ$.
We will then take the $CDF$ of $\mathcal{L}_{ABEZ}$ and $\mathcal{L}_{ADFZ}$ with our 
desired $t$ and average them out to get $P_t$ of $\mathcal{E}_2$. 

In short, given $paths$, $l_p$, and $l_q$, we will do the following:
\begin{enumerate}
    \item Split $paths$ into its individual path
    \item For each path, compute its latency distribution by convolving the latency distribution 
        $l_p$ and $l_q$ of each link
    \item With the resulting path latency distribution, determine the probability of temporal 
        property by computing the statistical property of said distribution
    \item Combine the temporal property of each path by computing the weighted average, based 
        on the load-balancing scheme of the control plane
\end{enumerate}

We do each of these steps to every Equivalence Class in $\mathcal{T}$, and do a sum-product operation 
of $P_f$ and $P_t$ to get the final probability of the property in question.

At this point, by introducing the temporal verification algorithm, we're essentially adding additional 
overhead to the exploration algorithm (on top of functional verification) that scales to the size of the 
Exploration Tree.
In other words, for $\mathcal{T}$ of size $n$, we will need to do temporal verification in each of 
those $n$ Equivalence Class.
We aim to minimize the overhead of temporal verification by introducing some optimization algorithms
later.

\subsection{Numeric Convolution}
There is one major problem with a convolution-based technique for computing the latency distribution 
of a given path: not every distribution pair can be convolved analytically.
A closed-form solution of a convolution is usually only available for two distributions of the same type.

In our model however, we intend to use real-world empirical data to be used as latency distribution, which 
might be arbitrarily shaped.
Therefore, in order to convolve two arbitrary distributions, we need to leverage a numerical convolution 
method.

We leverage an existing numerical convolution algorithm, DIRECT \cite{direct}, to fulfill this role.
We choose this method due to their bounded error property: DIRECT guarantee that the computed 
distribution and the correct theoretical distribution has a KL-divergence below a certain bound.
DIRECT has also been implemented in R's popular bayesmeta package \cite{bayesmeta}.
%TODO: cite and change font?

DIRECT works by numericaly approximating \textit{mixture distributions} from a large mixture components.
It is done by ``binning'' the mixture component / mixing distribution and intelligently chose a single 
reference value for each bin to compute the final CDF of the resulting mixing distribution.
The amount of bins and the reference value is determined based on the error bound that the user has set.

One subtle detail about DIRECT is while convolution is defined to be a commutative operation, and 
DIRECT also achieves this property, the order of operation matters for the algorithm's runtime 
performance.
In particular, if we had a chain of DIRECT convolutions, the result of a convolution should not be set 
as the second input distribution in the later convolutions to avoid performance penalty.
This is caused by a nested loop from the following two mechanisms:
\begin{itemize}
    \item DIRECT works by computing a list of support values by doing many PDF queries of the second input
        distribution
    \item Computing the PDF of a DIRECT distribution (the returned distribution of a DIRECT 
        algorithm) itself involves iterating over the current support values
\end{itemize}

As an ilustration, if we had 3 random variables $A$, $B$, and  $C$, and we want 
to convolve them all together, it would be faster to compute $direct(direct(A, B), C)$ instead of \\
$direct(C, direct(A, B))$. Trivially but importantly, we swap the input parameter if the latter is detected as a 
DIRECT distribution.

\section{Evaluation}

We implemented Tempus in $\sim600$ lines of Julia \cite{julia} code.
This code will take a topology configuration, source and destination node, and the 
temporal property in question and its threshold.
It will then output the final probability of the network upholding that temporal property.
We ran our prototype on a machine with 2 Intel E5-2630 v3 8-core CPU running up to 2.4 GHz 
and 128 GB of ECC RAM, provided by CloudLab \cite{cloudlab}.

To appraise the viability of our approach, we seek to answer 5 evaluation questions:
\begin{enumerate}
    \item How long does it take to run the verifier overall? How does it scale?
    \item What is the bottleneck step in the verification process?
    \item How effective is the optimization technique in reducing the verification time?
    \item How effective is the optimization technique in reducing the explored equivalence classes?
    \item How does different topologies affect optimization effectiveness?
\end{enumerate}

\subsection{Experiment Setup}
We tested Tempus on 5 real-world WAN topologies from the Topology Zoo and 3 datacenter topologies 
(fat-tree, with various amount of pods).
%TODO: stats of the topologies

\subsubsection{Functional Verification}
As verifying the routing protocol functional behavior is not the primary contribution of our research, 
we simply use OSPF as the routing protocol in our experiment with uniform link weights of 1.
We set 0.1\% failure rate in all links, in accordance to the failure rate in previous studies. %TODO: cite
We then run the functional verification with $10^{-8}$ inaccuracy level.

In order for our evaluation to be realistic, we set the source and destination node to be one of the 
edge routers (node with the smallest degree / smallest centrality?).
%TODO: edge-to-edge routers

\subsubsection{Temporal Verification}
For latency distribution, we show that we can use an empirical measurement as a distribution by 
using the reported queue length from DCTCP \cite{dctcp}.
By multiplying the queue length distribution with the line-rate, we could approximate the queuing 
latency distribution for a given router. 

Since the CDF operation that we define for temporal property verification is relatively cheap, 
we expect our result to generalize with any threshold we chose. 
In other words, while changing the threshold will change the final probability, it won't change 
the runtime performance of Tempus.
Thus, we set an arbitrary threshold for our evaluation.

\subsection{Runtime Profiling}

\begin{figure}[h]
    \centering
    \includegraphics[scale=0.5]{scalability}
    \caption{Runtime performance of the verification process, red dots represent the number of convolutions}
    \label{fig:scalability}
\end{figure}

\begin{figure}[h]
    \centering
    \includegraphics[scale=0.5]{scalability_fattree}
    \caption{Fat Tree, red dots represent the number of convolutions}
    \label{fig:scalabilityfat}
\end{figure}

\subsubsection{Performance and Scalability}
To begin our evaluation, we first measured the running duration of the verification process (and its steps) 
on various topologies.
Our results in Fig. \ref{fig:scalability} shows that our verification technique finished within a reasonable timeframe.
The verifier finished in the order of minutes, even for topologies with hundred of nodes.

From the same result, we also note that our verifier scales gracefully over the size of the network.
The verification duration ranges from 6 seconds to around 15 minutes. 
% TODO: fat-tree scalability
To drive this point more precisely, we also evaluate the running time of fat-tree topology in various sizes.
Our results in Fig. \ref{fig:scalabilityfat} shows that by keeping the same type of topology and scaling them up, we increased 
our verification time almost linearly, while the unoptimized version timed out after 2 hours.

\subsubsection{Bottleneck}
By looking at the proportion of time spent on each step in Fig. \ref{fig:scalability}, we could see that 
the combined \textbf{convolution procedure is the bottleneck} step in our verifier.
The convolution procedure takes 95\% - 99\% of the duration of the overall verification.
Hence we can see in the same figure that the runtime performance of Tempus and the total amount 
of convolution is highly correlated.

We conclude that for a reasonably low imprecision level, \textbf{Tempus runs in the order of 
minutes} and the performance of Tempus is primarily \textbf{bottlenecked by the total convolution 
operations}.

\subsection{Optimization Effectiveness}

\begin{figure}[h]
    \centering
    \includegraphics[scale=0.5]{optimization}
    \caption{Amount of convolution (in log scale) depending on what optimization strategy is applied}
    \label{fig:opt}
\end{figure}

We have established that the convolution procedure is a relatively expensive operation within 
our verifier.
Next, we inspected the effectiveness of our optimization techniques in reducing them.
To do that, we measured the overall running duration of the verifier while selectively disabling the 
optimization.

Our results in Fig. \ref{fig:opt} shows the effect of the two optimization techniques we introduced -- 
consolidation and memoization -- in reducing the verification duration and amount of convolution (both in log scale).
For baseline, we ran the verifier without any optimization, resulting in the left bar.
The middle bar represents the result where we enable only the consolidation strategy.
Finally, the right bar represents the result where we enable both the consolidation and memoization strategy.

We note that for most of the topologies, the combination of both optimization strategies resulted in 
\textbf{79\% - 98\% improvement in performance}.
Compared to the baseline performance, the consolidation strategy contributes to 30\% - 94\% of the improvement and
the memoization strategy contributes to 67\% - 96\% on top of that. 
%TODO: factors of the network node-pair? path amount and length?

We conclude that \textbf{consolidation and memoization are both effective optimization strategies} in our verification
framework.

\subsection{Equivalence Class Reduction}

\begin{figure}[h]
    \centering
    \includegraphics[scale=0.5]{ec}
    \caption{Amount of temporal EC compared to functional EC, label represents ratio}
    \label{fig:ec}
\end{figure}

While we cannot directly compare Tempus with other verifiers (due to difference in verification goal), 
we could use the amount of equivalence class as an indirect proxy of the verifier's behavior and performance.
Therefore, we measured the amount of additional equivalence classes introduced by Tempus in order to 
indirectly compare its overhead in addition to its functional counterparts.
This result is only influenced by the consolidation strategy, since the memoization strategy operates on 
a smaller granularity than equivalence classes.

Our results in Fig. \ref{fig:ec} shows the ratio between the amount of temporal equivalence classes that is 
being re-explored and the amount of functional equivalence class.
We note that we only need to \textbf{re-explore 4\% to 46\% equivalence classes} as an additional overhead 
compared to functional verification.

We conclude that \textbf{consolidation is effective in reducing the amount of equivalence classes that 
needs to be re-explored} in our verification framework.

\subsection{Optimization Analysis}

Up to this point, we have demonstrated that consolidation and memoization are two effective strategies in 
reducing the verification duration and amount of convolution procedure that needs to be done.
However, the results also show that their effectiveness varies depending on the topology.

To examine the factors that affects the effectiveness of each optimization strategy, we compute some properties 
of the topology that relates to the nature of the optimization.

% \begin{figure}[h]
%     \centering
%     \includegraphics[scale=0.5]{distribution}
%     \caption{The effect of distribution type on runtime performance of AttMpls, label represents the 
%     performance in seconds and red dots represents the amount of \textit{numerical} convolution}
%     \label{fig:dist}
% \end{figure}

\begin{itemize}
    \item While the amount of convolution gave us a pretty good proxy for performance, we could see 
        that there's still some variations
    \item This is due to the fact that we used two kinds of convolutions: analytical and numerical.
    \item Since numerical convolution is slower, we could see that for the same topology and the same amount of 
        overall convolution, \textbf{more numerical convolution will result in longer runtime performance}.
\end{itemize}

\begin{itemize}
    \item While the amount of path in those states can explain the amount of EC reduction, to get to 
        the amount of convolution, we also need to consider the length of each of those path
    \item By considering the both amount of path and its length to counting the convolution, we could 
        compare consolidation and memoization strategy to the unoptimized version
\end{itemize}

\begin{itemize}
    \item Compare the amount of temporal EC to the functional (unconsolidated) EC
    \item Corresponds to the consolidation strategy
    \item \textbf{More than 50\% reduction}
    \item Doesn't work equally on all topologies, depend on 
    \begin{itemize}
        \item How many functional EC are explored (for a given inaccuracy level)
        \item How many possible paths a given node pair have in those ECs
    \end{itemize}
\end{itemize}

% On the correctness side, we want to validate whether the result of our verifier is equivalent to 
% another verifier that has lower fidelity (i.e. functional property only) if we set the temporal 
% property to be virtually unconstrained (e.g. bounded reachability with a very high $T$).

% On the performance side, we will evaluate the verifier by measuring the amount of additional 
% overhead that the temporal verifier would introduce, measured by the amount of states that 
% need to be re-explored (in the second step) and / or the amount of convolution operation performed. 
% Early result on Highwinds \cite{knight2011internet} network shows that out of 2344 states 
% produced by step 1, only 459 got re-explored on step 2.

\section{Related Works}
\textbf{Deterministic Data and Control Plane Verifier}
There has been a substantial body of work concerning the functional behavior of routers
with a given Forwarding Information Base (FIB) or Routing Information Base (RIB).

Prior works like HSA \cite{hsa}, NetKAT \cite{netkat}, and VeriFlow \cite{veriflow} focuses on verifying the 
current data plane forwarding behavior against some property.
While useful in some real-time cases, these tools are not built to explore the many
possible forwarding table given an enumeration of failure cases.

Control plane verifiers such as ARC \cite{arc}, Minesweeper \cite{minesweeper}, and 
Tiramisu \cite{tiramisu} focuses on verifying the various data plane states that 
could be produced by a control plane configuration over multiple failure scenarios.
Many researchers has built on top of this idea, such as DNA \cite{dna} where they 
focuses on verifying the change in property in the event of configuration update.

As comprehensive as it is, both approaches only considers the functional behavior 
of the network (e.g. reachability, loop detection) and they traditionally verify 
the stated properties in a black or white fashion, only determining whether a given 
property is violated or not.

As a result, unlike Tempus, its usefulness does not extend into the use cases of 
verifying SLA, where such binary guarantees are often rightfully avoided in favor 
of a probabilistic and/or quantitative agreement.

\textbf{Probabilistic Verifier}
More recent works have tried to address this very issue.
Works like ProbNetKAT \cite{probnetkat} (a probabilistic verifier based on NetKAT 
\cite{netkat}), NetDice \cite{steffen2020probabilistic}, and SRE \cite{sre} provides 
the user ability to define probabilistic failure model, where a given component 
of the network can fail with a given probability.

However, most of them still considers the functional property of the network as the 
main focus. 
While they are useful in verifying certain types of SLA (e.g. availability), they don't 
focus on more quantitative aspect of the network (e.g. bandwidth, latency).

\textbf{Quantitative Verifier}
Some recent works are an exception to this trend, however.
Works like QARC \cite{subramanian2020detecting} (a quantitative verifier based on 
ARC \cite{arc}) has developed a framework to verify one particular quantitative 
property: excess bandwidth on a certain link.
While QARC is a deterministic verifier, like Tempus, they focuses on a more quantitative 
property. 
Unlike Tempus however, this work is orthogonal to verifying latency and thus complementary 
to Tempus.

%TODO: ICNA paper?

\textbf{Network Calculus}
CCAC?


\input{7_future.tex}

\section{Conclusion}
\begin{itemize}
    \item Tempus is a scalable latency SLA verifier. It could probabilistically verify in the order of minutes
    \item Consolidation and memoization is an effective optimization strategy, although the exact magnitude of 
        its effectiveness will depend on the network topology, specifically on the amount of paths and their 
        length.
\end{itemize}

\bibliography{cit}
\bibliographystyle{ieeetr}
\end{document}