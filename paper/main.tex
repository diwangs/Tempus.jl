\documentclass[10pt,sigconf,letterpaper,anonymous,nonacm]{acmart}

\usepackage{amsfonts} 

\title{Tempus: Probabilistic Network Latency Verifier}
\author{WIP}

\begin{document}
\begin{abstract}
    Test
\end{abstract}

\maketitle

\section{Introduction}
As society develops further in the information age and gets more dependent on interconnected 
digital services, the task of making those services reliable is of paramount priority.
The network that underlies those services is no exception to this. 

Modern networks consisted of various distributed protocols such as OSPF and BGP that exchange 
routing information so that a packet can reach their intended destination efficiently, even in the 
event of a failure.
Due to their configuration intricacies however, these protocols are notoriously hard to get right. 
It is hard for a network engineer to reason about whether their configured network fulfilled some 
intended property in various possible states of the network. 
Thus, they are left between the choice of accepting the reality of Murphy's Law or getting a tool 
to help them in this task, namely, network verifier.

In recent years, there have been many work done in this area. 
Data plane verifiers work on a single snapshot of the forwarding table of a given network to 
verify a property. 
However, if we want to verify how a given routing protocol will react to a specific failure, 
control plane verifier might be used instead. 
A control plane verifier will efficiently search for failures that might cause change in the 
property in question.

Earlier control plane verifiers are formulated to answer questions about a given property 
deterministically, in other words, yes or no answer. 
For example, k-link failures.
While useful to some extent, this kind of models are sometimes too restrictive since network 
operators are usually able to tolerate failures for a small fraction of time, particularly when 
reasoning about compliance with Service Level Agreement (SLA) where, for example, a given network 
only needs to be operational 99.999\% of the time. 
This kind of probabilistic properties have been the focus of a more recent works
\cite{steffen2020probabilistic}.

The property itself can also be divided into two kinds. 
Quantitative properties of the network, like reachability, waypointing, and loop existence, are 
the common properties that are studied by most of the existing verifiers. 
More recent works, like QARC, has also explored qualitative properties like link bandwidth 
violation for a given traffic.

In this work, we're exploring the other side of the network performance metric: latency. 

\section{Proposed Approach}
We will start with the assertion that latency is a property that only make sense 
after connectivity between two nodes has been established. 
In other words, if two nodes in a network aren't connected (due to physical failure 
or ACL policy), then the latency between them will \textit{always} be infinite.
Because of this, we divide the problem of latency verification into two parts: 
verifying that two nodes are reachable (\textit{functional} property) and only if 
the functional property is fulfilled, we would verify whether the latency between 
two nodes fulfill some property (\textit{temporal} property)

\subsection{Topology Graph}
For functional property verification, previous work \cite{steffen2020probabilistic} 
has laid the way for verifying reachability between two nodes under failure in a 
quantifiable and efficient manner. 
In this framework, the physical network is encoded in an edge-labeled directed graph 
$G_T = (V_T, E_T)$ where $V_T$ represents the routers in the network and 
$E_T$ represents the connectivity between a pair of source and destination router. 
The function $r: E_T \rightarrow \{x \in \mathbb{R} \mid 0 \le x \le 1\}$ is the 
edge-label that represents the failure rate of a given connectivity link.

On top of this topology, we could define additional information that would be used
by a routing protocol to determine the valid path(s) between two routers 
$src, dst \in V_T$ given a particular link failure scenario. In OSPF for example, 
we define a function $w_{ospf}: E_T \rightarrow \mathbb{N}$ as the edge-label that 
represents the positive weight of a link.

\subsection{Latency Graph}
There are a lot of components in a network that may introduce some latency into 
a transmitting packet. 
The principal challenge for us, then, is to determine the appropriate level of abstraction; 
detailed enough in order to accomplish our verification goal in a correct and accurate manner but 
abstract enough to not introduce scalability problems. 
We settle on modeling two source of latency that we deem significant: propagation delay and 
queuing delay.

Propagation delay is the latency that is introduced by the links in the network. 
This delay is independent of the load in the system, i.e. no matter how many in-flight packets 
are transmitting, the introduced delay is relatively the same. 
Queuing delay is the latency that the queuing process in the node introduced. 
Unlike propagation delay, queuing delay might be dependent on load in the system, since the more 
packets there are in the queue, the more delay the node will introduce to a subsequent packets.

To take both of these forms of delay into account, we've decided to model the latency of each
components with a univariate continuous random variable.
For propagation delay, the semantic of this random variable is relatively straightforward: it is 
the distribution of latency that a given link will introduce.
For queuing delay however, this random variable represents the delay that a given queuing process 
will introduce, marginalized over various traffic pattern that a given network state might have 
resulted.

% TODO: change this to v_i
To convey this additional notion of latency distribution, we form a second edge-labeled directed 
graph (called a latency graph) $G_L = (V_L, E_L)$.
For each network node $v \in V_T$, we want to ``expand'' this node to be able to represent the 
inner working of the queuing process within each node. 
Each $v$ will be expanded into $v_{in}, v_{out1}, v_{out2}, ..., v_{outn} \in V_L$ where $n$ is 
the amount of outward neighbors that $v$ has in $G_T$. 
Every inward edge that $v$ has will get directed to $v_{in}$.
$v_{out}$ is used to represent each physical port that $v$ has and thus, each outward edge that 
$v$ has will be sourced from one of the $v_{out}$ instead.
We can then add additional edges $e_1, e_2, ..., e_n$ that connects from $v_{in}$ to each 
$v_{out}$.
While each $v_{out}$ represents a port, these edges represent the output queue connected to those 
ports.
Since we already convey the routing and failure information in $G_T$, we instead label each 
$e \in E_L$ with the random variable distribution that each of these components might introduce.

\subsection{Latency Verification}
From these two graphs, we then do the verification in the following way:

First, we will do the functional reachability verification similar to what NetDice have done. 
The difference, however, is that we also collect additional information from each state (e.g. 
convergent paths) in order to be used in the temporal verification step. 
After the functional verifier finished exploring all states / stopped at a given threshold, we 
then move to the next step.

Second, we will compute the total path latency distribution for each convergent path in the state.
This is done by doing the convolution operation over the latency distribution of each component 
in the path. 
Since not all convolutions can be calculated analytically, we implemented a numerical convolution 
method for mixture distribution, DIRECT, which is able to convolve two distribution with a 
KL divergence error bound.

Third, after we compute the total path latency distribution, we calculate the probability of a 
given latency property being true in that state. For example, bounded reachability property 
(whether a packet can traverse from a source to destination below some time unit $T$) can be 
computed by integrating the PDF from $0$ to $T$. 
We combine all of the temporal probability of each state with the functional probability we 
calculated in the first step to get the probability of a given temporal property being true, 
which is what we want.

\section{Proposed Evaluation}
\begin{itemize}
    \item cross-validation with NetDice and DIRECT implementation in R
    \item performance evaluation with topology-zoo + dummy latency distribution
    \item performance evaluation with latency measurement data
\end{itemize}

\bibliography{cit}
\bibliographystyle{ieeetr}
\end{document}